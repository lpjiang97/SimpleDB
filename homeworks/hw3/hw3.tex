\documentclass[10pt]{myland}
\usepackage{pbox}

%%%%%%%%%%%title%%%%%%%%%%%%%%%%%%%%%%%
\newcommand{\myname}{Linxing Preston Jiang}
\newcommand{\quarter}{Winter 2018}
\newcommand{\myhwname}{\textbf{CSE 444: Homework 3}}
%%%%%%%%%%%%%gauss%%%%%%%%%%%%%%%%%
\pgfmathdeclarefunction{gauss}{2}{%
  \pgfmathparse{1/(#2*sqrt(2*pi))*exp(-((x-#1)^2)/(2*#2^2))}%
}
%%%%%%%%%%%%%%%%%%%%%%%%%%%%FILE%%%%%%%%%%%%%%%%%%%%%%%%%%%%%%%%%%%%
\begin{document}
\begin{center}
	{\Large \myhwname{CSE 444: Homework 3}} \\
	\vspace{.05in}
    \myname{Linxing Preston Jiang}\quad\quarter{Winter 2018}\\
	\vspace{.05in}
    \today \\
\end{center}
\vspace{.15in} \hrule \vspace{0.5em}%

% #########################################################QUESTION 1
\section{Concurrency Control with Locking}
	\begin{enumerate}
		\item
		\begin{enumerate}[label=(\alph*)]
			\item $R_2(X), R_2(Y), \underline{W_2(Y)}, R_1(X), \underline{R_1(Y)}, W_1(X), C_1, ... (\text{the rest of
				Transaction 2})$
			\item $\underline{R_1(X)}, R_1(Y), \underline{W_1(X)}, \underline{R_2(X)}, R_2(Y), W_2(Y), C_1, ...
				(\text{the rest of Transaction 2})$
            \item $R_1(X), R_1(Y), \underline{W_1(X)}, R_2(X), R_2(Y), W_2(Y), R_2(X), R_2(Y), \underline{W_2(X)},
				\underline{C_1}, ... (\text{the rest of Transaction 2})$
		\end{enumerate}

		\item No. $R_0(A)$ is before $R_2(A)$ so Transaction $T_0$ needs to precede $T_2$. $R_2(B)$ is before $W_0(B)$
			so Transaction $T_2$ needs to precede $T_0$, forming a cycle in the precedence graph.

		\item See the following schedule
            \begin{center}
                \begin{tabularx}{\linewidth}{Y|Y|Y}
					Transaction $T_0$ & Transaction $T_1$ & Transaction $T_2$ \\ \hline
					$L_0(A), L_0(B)$ & & \\
					$R_0(A)$ & & \\
					$W_0(A), U_0(A)$ & & \\
					& $L_1(A)$ -- \color{red}{DENIED}& $L_2(A), R_2(A)$\\
					& & $W_2(A)$\\
					& $L_1(A)$ -- \color{red}{DENIED}& \\
					$R_0(B)$ &  & $L_2(B)$ -- \color{red}{DENIED}\\
					$W_0(B), U_0(B)$ &  & \\
					&  & $L_2(B), R_2(B), U_2(A)$\\
					&  & $W_2(B), U_2(B)$\\
					& $L_1(A), R_1(A)$ & \\
					& $L_2(B), R_1(B)$ & \\
					& $U_1(A), U_1(B)$ & \\
                \end{tabularx}
            \end{center}

		\item By only releasing all locks when the transaction is completed, we have both conflict-serializable and
			recoverable schedules. Also we can avoid cascading aborts.
	\end{enumerate}

\newpage
\section{Optimistic Concurrency Control}
    \begin{enumerate}
		\item See below
            \begin{center}
                \begin{tabularx}{\linewidth}{Y|Y|Y|Y|Y|Y}
                    $T_1$ & $T_2$ & $T_3$ & $T_4$ & $X$ & $Y$ \\ \hline
                    1 & 2 & 3 & 4 & \pbox{20cm}{RT = 0 \\ WT = 0 \\ C = true} & \pbox{20cm}{RT = 0 \\ WT = 0 \\ C = true}\\ \hline
					  & $R_2(X)$ & & & RT = 2 & \\ \hline
					$R_1(X)$ & & & & RT = 2 & \\ \hline
					  & $W_2(X)$ & & & \pbox{20cm}{WT = 2 \\ C = false} & \\ \hline
					  & & & $W_4(X)$ & \pbox{20cm}{WT = 4} & \\ \hline
					\pbox{20cm}{$W_1(X)$ \\ \color{red}{ABORT}} & & & & & \\ \hline
					& & \pbox{20cm}{$W_3(X)$ \\ \color{red}{DELAY}} &  &  & \\ \hline
					& & & \color{red}{ABORT} &  & \\ \hline
					& $R_2(Y)$ & $W_3(X)$ &  & WT = 3 & RT = 2\\ \hline
					& $W_2(Y)$ &  &  & & \pbox{20cm}{WT = 2 \\ C = false}  \\ \hline
					& & \pbox{20cm}{$R_3(X)$ \\ \color{red}{DELAY}}  &  & & \\ \hline
					& $C_2$ & $R_3(X)$ & & \pbox{20cm}{RT =2 \\ WT = 3 \\ C = true} & RT = 3\\ \hline
					& & $W_3(X)$ & & & WT = 3\\ \hline
					& & $C_3$ & & & \pbox{20cm}{RT =3 \\ WT = 3 \\ C = true} \\ \hline
                \end{tabularx}
            \end{center}

		\item See below
            \begin{center}
                \begin{tabularx}{\linewidth}{Y|Y|Y|Y|Y|Y|Y}
                    $T_1$ & $T_2$ & $T_3$ & $T_4$ & $X_0$ & $X_3$ & $X_4$ \\ \hline
                    1 & 2 & 3 & 4 & & &  \\ \hline
					$R_1(X)$ & & & & RT = 1 & &  \\ \hline
					& & $R_3(X)$  & & RT = 3 & &  \\ \hline
					& & $W_3(X)$  & & & CREATE &  \\ \hline
					& $R_2(X)$ & & & RT = 3& &  \\ \hline
					& & & $R_4(X)$ & & RT = 4 &  \\ \hline
					& \pbox{30cm}{$W_2(X)$ \\ \color{red}{ABORT}} & & & & &  \\ \hline
					& & & $W_4(X)$ & &  & CREATE  \\ \hline
                \end{tabularx}
            \end{center}
	\end{enumerate}
\end{document}