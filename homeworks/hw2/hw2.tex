\documentclass[10pt]{myland}
\usepackage{amsmath}


%%%%%%%%%%%%%%%%%%%%%%%%%%%%FILE%%%%%%%%%%%%%%%%%%%%%%%%%%%%%%%%%%%%
\begin{document}
\begin{center}
	{\Large CSE 444: Homework 2} \\
	\vspace{.05in} 
    Linxing Preston Jiang \quad Winter 2018 \\
	\vspace{.05in} 
    \today \\
\end{center}
\vspace{.15in} \hrule \vspace{0.5em}%

\section{B+ Trees}
\begin{enumerate}[label=\textbf{\arabic*.}, listparindent=0.0em, itemsep=1em]
    \item See below
    \begin{center}
    \scalebox{0.5}{
        \begin{tikzpicture}
            %
            \btreeinodefour{root}{80}{}{}{};
            \xyshift{-80mm}{-20mm}{\btreeinodefour{root1l}{19}{25}{30}{60}}
            \xyshift{40mm}{-20mm}{\btreeinodefour{root1r}{100}{120}{140}{}}
            \btreelink{root-1}{root1l}
            \btreelink{root-2}{root1r}
            \xyshift{-180mm}{-40mm}{\btreeinodefour{leafa}{10}{15}{18}{}}
            \xyshift{-130mm}{-40mm}{\btreeinodefour{leafb}{19}{20}{}{}}
            \xyshift{-80mm}{-40mm}{\btreeinodefour{leafc}{25}{26}{27}{}}
            \xyshift{-30mm}{-40mm}{\btreeinodefour{leafd}{40}{50}{}{}}
            \xyshift{20mm}{-40mm}{\btreeinodefour{leafe}{60}{65}{70}{75}}
            \xyshift{70mm}{-40mm}{\btreeinodefour{leaff}{80}{85}{90}{}}
            \foreach \x\y in {1/a, 2/b, 3/c, 4/d, 5/e} {\btreelink{root1l-\x}{leaf\y}}
            \btreelink{root1r-1}{leaff}
            \node (empty1) at (120mm, -40mm){};
            \node (empty2) at (130mm, -40mm){};
            \node (empty3) at (140mm, -40mm){};
            \path[btlink](root1r-2) edge (empty1);
            \path[btlink](root1r-3) edge (empty2);
            \path[btlink](root1r-4) edge (empty3);
            \xyshift{120mm}{-40mm}{}
            \path[btlink] (leafa-5) edge (leafb-1);
            \path[btlink] (leafb-5) edge (leafc-1);
            \path[btlink] (leafc-5) edge (leafd-1);
            \path[btlink] (leafd-5) edge (leafe-1);
            \path[btlink] (leafe-5) edge (leaff-1);
            %%% values
            \foreach \x/\y in {1/10, 2/15, 3/18} {
                \node[below = 10mm of leafa-\x, btreevale] (leafa\x) {\y};
                \path[btlink] (leafa-\x) edge (leafa\x);
            }
            \foreach \x/\y in {1/19, 2/20} {
                \node[below = 10mm of leafb-\x, btreevale] (leafb\x) {\y};
                \path[btlink] (leafb-\x) edge (leafb\x);
            }
            \foreach \x/\y in {1/25, 2/26, 3/27} {
                \node[below = 10mm of leafc-\x, btreevale] (leafc\x) {\y};
                \path[btlink] (leafc-\x) edge (leafc\x);
            }
            \foreach \x/\y in {1/40, 2/50} {
                \node[below = 10mm of leafd-\x, btreevale] (leafd\x) {\y};
                \path[btlink] (leafd-\x) edge (leafd\x);
            }
            \foreach \x/\y in {1/60, 2/65, 3/70, 4/75} {
                \node[below = 10mm of leafe-\x, btreevale] (leafe\x) {\y};
                \path[btlink] (leafe-\x) edge (leafe\x);
            }
            \foreach \x/\y in {1/80, 2/85, 3/90} {
                \node[below = 10mm of leaff-\x, btreevale] (leaff\x) {\y};
                \path[btlink] (leaff-\x) edge (leaff\x);
            }
        \end{tikzpicture}
    }
    \end{center}

    \item See below
    \begin{center}
    \scalebox{0.5}{
        \begin{tikzpicture}
            %
            \btreeinodefour{root}{80}{}{}{};
            \xyshift{-80mm}{-20mm}{\btreeinodefour{root1l}{40}{60}{}{}}
            \xyshift{40mm}{-20mm}{\btreeinodefour{root1r}{100}{120}{140}{}}
            \btreelink{root-1}{root1l}
            \btreelink{root-2}{root1r}
            \xyshift{-130mm}{-40mm}{\btreeinodefour{leafb}{18}{19}{20}{}}
            \xyshift{-70mm}{-40mm}{\btreeinodefour{leafd}{40}{50}{}{}}
            \xyshift{-10mm}{-40mm}{\btreeinodefour{leafe}{60}{65}{70}{75}}
            \xyshift{50mm}{-40mm}{\btreeinodefour{leaff}{80}{85}{90}{}}
            \foreach \x\y in {1/b, 2/d, 3/e} {\btreelink{root1l-\x}{leaf\y}}
            \btreelink{root1r-1}{leaff}
            \node (empty1) at (120mm, -40mm){};
            \node (empty2) at (130mm, -40mm){};
            \node (empty3) at (140mm, -40mm){};
            \path[btlink](root1r-2) edge (empty1);
            \path[btlink](root1r-3) edge (empty2);
            \path[btlink](root1r-4) edge (empty3);
            \xyshift{120mm}{-40mm}{}
            \path[btlink] (leafb-5) edge (leafd-1);
            \path[btlink] (leafd-5) edge (leafe-1);
            \path[btlink] (leafe-5) edge (leaff-1);
            %%% values
            \foreach \x/\y in {1/18, 2/19, 3/20} {
                \node[below = 10mm of leafb-\x, btreevale] (leafb\x) {\y};
                \path[btlink] (leafb-\x) edge (leafb\x);
            }
            \foreach \x/\y in {1/40, 2/50} {
                \node[below = 10mm of leafd-\x, btreevale] (leafd\x) {\y};
                \path[btlink] (leafd-\x) edge (leafd\x);
            }
            \foreach \x/\y in {1/60, 2/65, 3/70, 4/75} {
                \node[below = 10mm of leafe-\x, btreevale] (leafe\x) {\y};
                \path[btlink] (leafe-\x) edge (leafe\x);
            }
            \foreach \x/\y in {1/80, 2/85, 3/90} {
                \node[below = 10mm of leaff-\x, btreevale] (leaff\x) {\y};
                \path[btlink] (leaff-\x) edge (leaff\x);
            }
        \end{tikzpicture}
    }
    \end{center}

    \item
        \begin{itemize}
            \item We need to look up 4 pages: 3 pages of index files, 1 page which contains the key 40
            \item Same as above, 4 pages
            \item We need to look up 3 pages to get to the key 60, and then we can keep the next page until reached 80.
                Then go back to the index node to find where 90 is, then read 90. So 3 + 2 + 2 = \textbf{7} pages in
                total.
            \item We need to look up 3 pages to get to the key 60. Because we assume the worst case of unclustered index,
                everything is on different pages. So we need to read 4 pages until 80 to go back to the index node, then
                read 80, 85, 90 on three different pages. So 3 + 4 + 1 + 3 = \textbf{11} pages.
        \end{itemize}
\end{enumerate}

\section{Operator algorithms}

\begin{enumerate}[label=\textbf{\arabic*.}, listparindent=0.0em, itemsep=1em]
    \item
        \begin{itemize}
            \item $\text{Cost} = B(R) + T(R)\cdot B(S) = 100 + 1000\cdot 80 = 80100$
            \item $\text{Cost} = B(R) + B(R)\cdot B(S) = 100 + 100\cdot 80 = 8100$
        \end{itemize}
    \item The algorithm with the least cost would be block-nested-looped. The cost is (assume $S$ as the outer relation)
    \[\text{Cost} = B(S) + \frac{B(R)\cdot B(S)}{M - 2} = 80 + \frac{100\cdot 80}{10 - 2} = 1080\]
    \item The index-based nested loop join will iterate over R, for each tuple in R, fetch the corresponding tuples from
    S. If we assume there is exactly one match for every tuple in S, then for every tuple in R we make 1 disk access of
    S. So the total cost is
    \[B(R) + T(R) * 1 = B(R) + T(R) = 100 + 1000 = 1100\]
\end{enumerate}

\section{Multipass algorithms}
\begin{enumerate}[label=\textbf{\arabic*.}, listparindent=0.0em, itemsep=1em]
    \item Use partitioned hash-join. The steps are:
        \begin{enumerate}[label=\textbf{(\arabic*)}, listparindent=0.0em, itemsep=1em]
            \item Partition $R$ with a hash function $h$ that maps the tuple into ten different pages in $M$ (one page
                reserved for input buffer). Write the tuple to disk when the page is full. So in the end we have 10
                partitions for R on disk. Each has length 100 blocks (assume the evenly distributing hash function).
            \item Repeat (1) for $S$.
            \item The partitions right now won't fit in memory page. So we still need another hash function $h2$ to
                further partition the groups of $R$ now, the write the results to disk. So now we have 100 length 10 blocks
                partitions for $R$.
            \item Repeat (3) for $S$.
            \item Read 5 blocks from $R$ partition, 5 blocks from $S$ partition, reserve one page for output buffer, and
                join until exhaustion.
        \end{enumerate}
        Total cost: $5B(R) + 5(S) = 9000$
    \item User sorted-merge-join. The steps are:
        \begin{enumerate}[label=\textbf{(\arabic*)}, listparindent=0.0em, itemsep=1em]
            \item Read $M$ blocks from $R$ into memory and sort them, then write them out to disk. Repeat until $R$ is
                exhausted. This produces 91 runs in disk of $R$.
            \item Repeat (1) for $S$. This produces 73 runs in disk of $S$.
            \item For every $M - 1$ runs in disk of $R$, merge them memory to produce $\left\lceil\frac{91}{11}\right
                \rceil = 10$ longer sorted runs.
            \item Repeat (3) to produce one final sorted run of $R$ written to disk.
            \item Repeat (3) and (4) for the 73 runs in disk of $S$ to get one final sorted run of $S$.
            \item Read the two sorted runs in memory and join them.
        \end{enumerate}
        Total cost: $7B(R) + 7B(S) = 12,600$
\end{enumerate}
\end{document}
