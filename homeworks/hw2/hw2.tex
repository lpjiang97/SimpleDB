\documentclass[10pt]{myland}
\usepackage{amsmath}


%%%%%%%%%%%%%%%%%%%%%%%%%%%%FILE%%%%%%%%%%%%%%%%%%%%%%%%%%%%%%%%%%%%
\begin{document}
\begin{center}
	{\Large CSE 444: Homework 2} \\
	\vspace{.05in} 
    Linxing Preston Jiang \quad Winter 2018 \\
	\vspace{.05in} 
    \today \\
\end{center}
\vspace{.15in} \hrule \vspace{0.5em}%

\section{B+ Trees}
\begin{enumerate}[label=\textbf{\arabic*.}, listparindent=0.0em, itemsep=1em]
    \item See below
    \begin{center}
    \scalebox{0.5}{
        \begin{tikzpicture}
            %
            \btreeinodefour{root}{80}{}{}{};
            \xyshift{-80mm}{-20mm}{\btreeinodefour{root1l}{19}{25}{30}{60}}
            \xyshift{40mm}{-20mm}{\btreeinodefour{root1r}{100}{120}{140}{}}
            \btreelink{root-1}{root1l}
            \btreelink{root-2}{root1r}
            \xyshift{-180mm}{-40mm}{\btreeinodefour{leafa}{10}{15}{18}{}}
            \xyshift{-130mm}{-40mm}{\btreeinodefour{leafb}{19}{20}{}{}}
            \xyshift{-80mm}{-40mm}{\btreeinodefour{leafc}{25}{26}{27}{}}
            \xyshift{-30mm}{-40mm}{\btreeinodefour{leafd}{40}{50}{}{}}
            \xyshift{20mm}{-40mm}{\btreeinodefour{leafe}{60}{65}{70}{75}}
            \xyshift{70mm}{-40mm}{\btreeinodefour{leaff}{80}{85}{90}{}}
            \foreach \x\y in {1/a, 2/b, 3/c, 4/d, 5/e} {\btreelink{root1l-\x}{leaf\y}}
            \btreelink{root1r-1}{leaff}
            \node (empty1) at (120mm, -40mm){};
            \node (empty2) at (130mm, -40mm){};
            \node (empty3) at (140mm, -40mm){};
            \path[btlink](root1r-2) edge (empty1);
            \path[btlink](root1r-3) edge (empty2);
            \path[btlink](root1r-4) edge (empty3);
            \xyshift{120mm}{-40mm}{}
            \path[btlink] (leafa-5) edge (leafb-1);
            \path[btlink] (leafb-5) edge (leafc-1);
            \path[btlink] (leafc-5) edge (leafd-1);
            \path[btlink] (leafd-5) edge (leafe-1);
            \path[btlink] (leafe-5) edge (leaff-1);
            %%% values
            \foreach \x/\y in {1/10, 2/15, 3/18} {
                \node[below = 10mm of leafa-\x, btreevale] (leafa\x) {\y};
                \path[btlink] (leafa-\x) edge (leafa\x);
            }
            \foreach \x/\y in {1/19, 2/20} {
                \node[below = 10mm of leafb-\x, btreevale] (leafb\x) {\y};
                \path[btlink] (leafb-\x) edge (leafb\x);
            }
            \foreach \x/\y in {1/25, 2/26, 3/27} {
                \node[below = 10mm of leafc-\x, btreevale] (leafc\x) {\y};
                \path[btlink] (leafc-\x) edge (leafc\x);
            }
            \foreach \x/\y in {1/40, 2/50} {
                \node[below = 10mm of leafd-\x, btreevale] (leafd\x) {\y};
                \path[btlink] (leafd-\x) edge (leafd\x);
            }
            \foreach \x/\y in {1/60, 2/65, 3/70, 4/75} {
                \node[below = 10mm of leafe-\x, btreevale] (leafe\x) {\y};
                \path[btlink] (leafe-\x) edge (leafe\x);
            }
            \foreach \x/\y in {1/80, 2/85, 3/90} {
                \node[below = 10mm of leaff-\x, btreevale] (leaff\x) {\y};
                \path[btlink] (leaff-\x) edge (leaff\x);
            }
        \end{tikzpicture}
    }
    \end{center}

    \item See below
    \begin{center}
    \scalebox{0.5}{
        \begin{tikzpicture}
            %
            \btreeinodefour{root}{80}{}{}{};
            \xyshift{-80mm}{-20mm}{\btreeinodefour{root1l}{40}{60}{}{}}
            \xyshift{40mm}{-20mm}{\btreeinodefour{root1r}{100}{120}{140}{}}
            \btreelink{root-1}{root1l}
            \btreelink{root-2}{root1r}
            \xyshift{-130mm}{-40mm}{\btreeinodefour{leafb}{18}{19}{20}{}}
            \xyshift{-70mm}{-40mm}{\btreeinodefour{leafd}{40}{50}{}{}}
            \xyshift{-10mm}{-40mm}{\btreeinodefour{leafe}{60}{65}{70}{75}}
            \xyshift{50mm}{-40mm}{\btreeinodefour{leaff}{80}{85}{90}{}}
            \foreach \x\y in {1/b, 2/d, 3/e} {\btreelink{root1l-\x}{leaf\y}}
            \btreelink{root1r-1}{leaff}
            \node (empty1) at (120mm, -40mm){};
            \node (empty2) at (130mm, -40mm){};
            \node (empty3) at (140mm, -40mm){};
            \path[btlink](root1r-2) edge (empty1);
            \path[btlink](root1r-3) edge (empty2);
            \path[btlink](root1r-4) edge (empty3);
            \xyshift{120mm}{-40mm}{}
            \path[btlink] (leafb-5) edge (leafd-1);
            \path[btlink] (leafd-5) edge (leafe-1);
            \path[btlink] (leafe-5) edge (leaff-1);
            %%% values
            \foreach \x/\y in {1/18, 2/19, 3/20} {
                \node[below = 10mm of leafb-\x, btreevale] (leafb\x) {\y};
                \path[btlink] (leafb-\x) edge (leafb\x);
            }
            \foreach \x/\y in {1/40, 2/50} {
                \node[below = 10mm of leafd-\x, btreevale] (leafd\x) {\y};
                \path[btlink] (leafd-\x) edge (leafd\x);
            }
            \foreach \x/\y in {1/60, 2/65, 3/70, 4/75} {
                \node[below = 10mm of leafe-\x, btreevale] (leafe\x) {\y};
                \path[btlink] (leafe-\x) edge (leafe\x);
            }
            \foreach \x/\y in {1/80, 2/85, 3/90} {
                \node[below = 10mm of leaff-\x, btreevale] (leaff\x) {\y};
                \path[btlink] (leaff-\x) edge (leaff\x);
            }
        \end{tikzpicture}
    }
    \end{center}


    \item
        \begin{itemize}
            \item We need to look up 4 pages: 3 pages of index files, 1 page which contains the key 40
            \item Same as above, 4 pages
            \item We need to look up 4 pages to get to the key 60, and then we can keep reading from the page until
                crossing to the next page with key 90 so \textbf{5} pages in total.
            \item We need to look up 4 pages to get to the key 60, and then we can keep reading until the end of the
                page. Then we need to go the next \textbf{index} file to find key 80, then read from the pointed data
                file until 90 is read. So \textbf{6} pages in total.
        \end{itemize}
\end{enumerate}


\section{Operator algorithms}

\begin{enumerate}[label=\textbf{\arabic*.}, listparindent=0.0em, itemsep=1em]
    \item
        \begin{itemize}
            \item $\text{Cost} = B(R) + T(R)\cdot B(S) = 100 + 1000\cdot 80 = 80100$
            \item $\text{Cost} = B(R) + B(R)\cdot B(S) = 100 + 100\cdot 80 = 8100$
        \end{itemize}
    \item The algorithm with the least cost would be block-nested-looped. The cost is
    \[\text{Cost} = B(R) + \frac{B(R)\cdot B(S)}{M - 1} = 100 + \frac{100\cdot 80}{10 - 1} = \frac{8900}{9}\]
    \item The index-based nested loop join will iterate over R, for each tuple in R, fetch the corresponding tuples from
    S. If we assume there is exactly one match for every tuple in S, then for every tuple in R we make 1 disk access of
    S. So the total cost is $B(R) + T(R) * 1 = B(R) + T(R)$
\end{enumerate}

\end{document}
