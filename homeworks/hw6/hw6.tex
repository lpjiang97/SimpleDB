\documentclass[10pt]{myland}
\usepackage{amsmath}
\usepackage{relsize}

%bowtie sign
\let\oldtriangleleft\triangleleft
\renewcommand\triangleleft{\mathbin{\color{green}\oldtriangleleft}}
\DeclareRobustCommand\bowtie{\mathrel\triangleright\joinrel\mathrel\oldtriangleleft}

%%%%%%%%%%%%%%%%%%%%%%%%%%%%FILE%%%%%%%%%%%%%%%%%%%%%%%%%%%%%%%%%%%%
\begin{document}
\begin{center}
	{\Large CSE 444: Homework 6} \\
	\vspace{.05in} 
    Linxing Preston Jiang \quad Winter 2018 \\
	\vspace{.05in} 
    \today \\
\end{center}
\vspace{.15in} \hrule \vspace{0.5em}%

\section{Parallel Data Processing}
\begin{enumerate}[label=\textbf{\arabic*.}, listparindent=0.0em, itemsep=1em]
    \item See below
    \begin{center}
    \scalebox{0.6}{
        \begin{tikzpicture}[
            >=latex,shorten >=2pt,shorten <=2pt,shape aspect=1,
        ]
            \tikzstyle{cyl} = [cylinder, shape border rotate=90, draw, minimum height=1cm, minimum width=5cm]
            \tikzstyle{noo} = [rectangle, draw, minimum height=1cm, minimum width=5cm]
            \tikzstyle{cir} = [ellipse, draw, minimum height=1cm, minimum width=2cm]
            \tikzstyle{joining} = []

            \node[cyl](db1) at (-10, 0) {$\frac{1}{3} $of $R$ and $S$};
            \node[cyl](db2) at (0, 0) {$\frac{1}{3} $of $R$ and $S$};
            \node[cyl](db3) at (10, 0) {$\frac{1}{3} $of $R$ and $S$};
            \node[noo](node1) at (-10, 3) {Node 1};
            \node[noo](node2) at (0, 3) {Node 2};
            \node[noo](node3) at (10, 3) {Node 3};
            \node[cir] (scan11) at (-12, 4) {Scan $R$};
            \node[cir] (scan12) at (-8, 4) {Scan $S$};
            \node[cir] (scan21) at (-2, 4) {Scan $R$};
            \node[cir] (scan22) at (2, 4) {Scan $S$};
            \node[cir] (scan31) at (8, 4) {Scan $R$};
            \node[cir] (scan32) at (12, 4) {Scan $S$};
            \node[cir] (sel1) at (-8, 6) {$\sigma_{S.d > 0}$};
            \node[cir] (sel2) at (2, 6) {$\sigma_{S.d > 0}$};
            \node[cir] (sel3) at (12, 6) {$\sigma_{S.d > 0}$};
            \node[cir] (gc1) at (-8, 8) {$\gamma_{S.c, \text{sum(d)}\to s, \text{count(d)}\to c}$};
            \node[cir] (gc2) at (2, 8) {$\gamma_{S.c, \text{sum(d)}\to s, \text{count(d)}\to c}$};
            \node[cir] (gc3) at (12, 8) {$\gamma_{S.c, \text{sum(d)}\to s, \text{count(d)}\to c}$};
            \node[cir] (hb1) at (-12, 8) {hash on $R.b$};
            \node[cir] (hb2) at (-2, 8) {hash on $R.b$};
            \node[cir] (hb3) at (8, 8) {hash on $R.b$};
            \node[joining] (j1) at (-10, 10) {\larger[3]$\bowtie_{R.b = S.c}$};
            \node[joining] (j2) at (0, 10) {\larger[3]$\bowtie_{R.b = S.c}$};
            \node[joining] (j3) at (10, 10) {\larger[3]$\bowtie_{R.b = S.c}$};
            \node[cir] (ga1) at (-10, 12) {$\gamma_{R.a, \text{sum(s)}\to s_1, \text{sum(c)}\to c_1}$};
            \node[cir] (ga2) at (0, 12) {$\gamma_{R.a, \text{sum(s)}\to s_1, \text{sum(c)}\to c_1}$};
            \node[cir] (ga3) at (10, 12) {$\gamma_{R.a, \text{sum(s)}\to s_1, \text{sum(c)}\to c_1}$};
            \node[cir] (ha1) at (-10, 14) {hash on $R.a$};
            \node[cir] (ha2) at (0, 14) {hash on $R.a$};
            \node[cir] (ha3) at (10, 14) {hash on $R.a$};
            \node[cir] (ans1) at (-10, 16) {$\gamma_{R.a, \text{sum}(s1)/\text{sum(c1)}\to\text{avg}(d)}$};
            \node[cir] (ans2) at (0, 16) {$\gamma_{R.a, \text{sum}(s1)/\text{sum(c1)}\to\text{avg}(d)}$};
            \node[cir] (ans3) at (10, 16) {$\gamma_{R.a, \text{sum}(s1)/\text{sum(c1)}\to\text{avg}(d)}$};
            % link nodes
            \foreach \n in {1, 2, 3} {
                \path (scan\n2) edge (sel\n);
                \path (scan\n1) edge (hb\n);
                \path (sel\n) edge (gc\n);
                \path (gc\n) edge (j\n);
                \path (j\n) edge (ga\n);
                \path (ga\n) edge (ha\n);
            }
            \foreach \x in {1, 2, 3} {
                \foreach \y in {1, 2, 3} {
                    \path (hb\x) edge (j\y);
                    \path (ha\x) edge (ans\y);
                }
            }
        \end{tikzpicture}
    }
    \end{center}
    \item We will MapReduce twice:
        \begin{itemize}
            \item First round:
                \begin{itemize}
                    \item Map: Map function on $R$: Use $R.b$ as key, write value as $('R', R.a)$; map function on $S$:
                        First apply filter on $S.d > 0$, use $S.c$ as key, write value as $('S', S.d)$
                    \item Reduce: The same value of $R.b$ and $S.c$ as the key, perform a local join and output value is
                        $(R.a, S.d)$
                \end{itemize}
            \item Second round (MapReduce on $(R.a, S.d)$):
                \begin{itemize}
                    \item Map: Use $R.a$ as the key, output value as $(S.d)$
                    \item Reduce: Input is now $S.d$ with the same value of $R.a$. Output $R.a$ and $\frac{sum(S.d)}{count(S.d)}$
                        as the average.
                \end{itemize}
        \end{itemize}
\end{enumerate}

\section{Distribution and Replication}
\begin{enumerate}
    \item Subordinate will scan the log file and find there is PREPARE by no COMMIT/ABORT. It will keep asking the
        coordinator for a final decision. If it's commit, we redo the transaction. If it's abort, we undo the transaction
    \item
        \begin{itemize}
            \item Single master
                \begin{itemize}
                    \item Asynchronus approach has better availability because it does not need to update all replicas
                        together at the same time
                    \item Asynchronus approach has worse consistency because when the master of asynchronous approach
                        fails, it might lose some recent write updates which haven't be sent to replicas
                \end{itemize}
            \item Multi masters
                \begin{itemize}
                    \item Has the same better availability as an asynchronus approach, also allows more than one
                        transactions to run together.
                    \item Asynchronus approach may introduce conflicts when different transactions write to different
                        values of the same object on multiple replicas. Need conflicts detection and resolution.
                \end{itemize}
        \end{itemize}
\end{enumerate}
\end{document}
