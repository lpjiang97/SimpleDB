\documentclass[10pt]{myland}

\renewcommand{\tikzmark}[1]{\tikz[overlay,remember picture] \node (#1) {};}
\newcommand{\DrawLine}[3][]{%
    \begin{tikzpicture}[overlay,remember picture]
        \draw [#1] ($(#2)+(-5ex,0.6ex)$) -- ($(#3)+(5ex,0.6ex)$);
    \end{tikzpicture}%
}%

\newcommand{\DrawRect}[4][]{%
    \begin{tikzpicture}[overlay,remember picture]
        \draw [#1] ($(#2)+(-9ex,-2.6ex)$) rectangle ($(#3)+(11ex,3.2ex)$);
        \node [#1, draw] at ($(#3)+(15ex, 0.6ex)$) {#4};
    \end{tikzpicture}%
}%

%%%%%%%%%%%%%%%%%%%%%%%%%%%%FILE%%%%%%%%%%%%%%%%%%%%%%%%%%%%%%%%%%%%
\begin{document}

\begin{center}
	{\Large \myhwname{CSE 444: Homework 5}} \\
	\vspace{.05in}
    \myname{Linxing Preston Jiang}\quad\quarter{Winter 2018}\\
	\vspace{.05in}
    \today \\
\end{center}
\vspace{.15in} \hrule \vspace{0.5em}%

\section{Query Plan Cost Computation}
\begin{enumerate}[label=\textbf{\arabic*.}, listparindent=0.0em, itemsep=1em]
  \item 
    \begin{enumerate}[label=\textbf{(\alph*)}, listparindent=0.0em, itemsep=1em] 
      \item $\frac{1000 - 100}{9000 - 0} = \frac{1}{10}$
      \item $\frac{1}{V(S, e)}\cdot\frac{1}{V(S, f)} = \frac{1}{10}\cdot\frac{1}{100} = \frac{1}{1000}$
      \item $\frac{1}{\max\{V(R, c), V(S, d)\}} = \frac{1}{50}$
      \item $\frac{1}{V(R, b)} + \frac{1}{V(R, b)} = \frac{1}{50}$
      \item $\frac{1}{\max\{V(S, g), V(T, h)\}} = \frac{1}{100}$
    \end{enumerate}
  \item 
    \begin{itemize}
      \item $|R_1| = 10000\cdot 0.1 = 1000$
      \item $|R_2| = 10000\cdot\frac{1}{1000} = 10$
      \item $|R_3| = \frac{1000\cdot 10}{50} = 200$
      \item $|R_4| = \frac{200}{50} = 4$
      \item $|R_5| = \frac{4\cdot 10000}{100} = 400$
      \item $|R_6| = |R_5| = 400$
    \end{itemize}
  \item 
    \begin{enumerate}[label=\textbf{(\alph*)}, listparindent=0.0em, itemsep=1em] 
      \item Because of the clustered index on $a$, we have $B(R)\cdot 0.10 = 100$ IOs to read from R. And the unclustered index 
        on (e, f) will help reduce the number of IOs to $B(S)\cdot \frac{1}{1000} = 10$ pages. We do not write intermediate results
        to disk, so the next IO cost comes from reading from $T$, which will have $T(R_4) \cdot \frac{B(T)}{V(T, h)} = 4\cdot 
        \frac{1000}{100} = 40$. There the total cost is 
        \[100 + 10 + 40 = 150\]
    \end{enumerate}

\end{enumerate}

\newpage
\section{Query Optimization}
\begin{enumerate}[label=\textbf{\arabic*.}, listparindent=0.0em, itemsep=1em]
  \item See table \par
    \begin{tabularx}{\linewidth}{Y|Y|Y|Y|Y}
			\hline
      \textbf{Subqeury} & \textbf{Cost} & \textbf{Size of ouptut} & \textbf{Plan} & \textbf{P/K} \\ \hline
      R & 100 page IOs & 1K records on 100 pages & Sequential scan of R & K \\ \hline
      S & 1K page IOs & 10K records on 1K pages & Sequential scan of S & K \\ \hline
      W & 10 page IOs & 100 records on 10 pages & Sequential scan of W & K \\ \hline
      RS & 100 + 100 * 1000 = 100100 page IOs & 10K records on 2K pages & Nested loop join of R (outer) S & K \\ \hline
      RW & 100 + 100 * 10 = 1100 page IOs & 100 records on 20 pages & Nested loop join of R (outer) W & P (WR has lower IO cost) \\ \hline
      SR & 1000 + 1000 * 100 = 101000 page IOs & 10K records on 2K pages & Nested loop join of S (outer) R & P (RS has lower IO cost) \\ \hline
      WR & 10 + 10 * 100 = 1010 page IOs &  100 records on 20 pages & Nested loop join of W (outer) R & K \\ \hline
      \tikzmark{StartA}SW & ... & ... & ... & P \tikzmark{EndA}\\\hline 
      \tikzmark{StartB}WS & ... & ... & ... & P \tikzmark{EndB}\\\hline 
      RSW & 100100 + 2K * 10  = 120,100 & 1K records on 300 pages & Nested loop join of RS (outer) W & P \\ \hline
      \tikzmark{StartC}WRS & 1010 + 20 * 1K = 21,010 & 1K records on 300 pages & Nested loop join of WR (outer) S & K \tikzmark{EndC} \\ \hline
		\end{tabularx}
    \DrawLine[red, thick]{StartA}{EndA}
    \DrawLine[red, thick]{StartB}{EndB}
    \begin{tikzpicture}[overlay,remember picture]
        \node[draw, red] at ($(EndA)+(14ex,-1ex)$) {\small{ Cartesian Product }};
    \end{tikzpicture}%
    \DrawRect[red, thick]{StartC}{EndC}{Best}
    \par The physical plan is shown here: \par
    \begin{center}
    \begin{tikzpicture}[
      scale=.6,
    ]
      \node (W) at(-20, 0) {\texttt{W}};
      \node[right=3cm of W] (R) {\texttt{R}};
      \node[above right=1.5cm and 1.5cm of W] (WR) {$\bowtie_{\texttt{R.c = W.h}}$};
      \node[draw, left=0.6cm of WR] (WRtext) {Nested Loop Join};
      \path (W) edge (WR);
      \path (R) edge (WR);
      \node[above right=1.5cm and 1.5cm of WR] (WRS) {$\bowtie_{\texttt{R.a = S.d}}$};
      \node[right=3cm of R] (S) {\texttt{S}};
      \path (WR) edge (WRS);
      \path (S) edge (WRS);
      \node[draw, left=0.6cm of WRS] (WRStext) {Nested Loop Join};
      \node[above=1.5cm of WRS] (final) {$\pi_{*}$};
      \path (WRS) edge (final);
    \end{tikzpicture}%
    \end{center}
  %%%%%
  \item 
    \begin{itemize}
      \item Clustered index on \texttt{R.b}: Because of the selection \texttt{R.b > 100}, this index will help reduce the number of reads from 
        \texttt{R} by the selectivity of \texttt{R.b > 100}.
       \item Unclusted index on \texttt{S.d}: It depends on the estimated number of tuples from \texttt{R} to be read. Because we are joining 
        \texttt{R.a} and \texttt{S.d}, if the number of tuples from \texttt{R} is more than the number of pages from \texttt{S}, we should
        just perform a file scan because every read from an unclustered index can be on a new page. Otherwise, we can choose to use this 
        unclustered index to reduce the number of reads from \texttt{S}.
    \end{itemize}
  %%%%%
  \item An interesting order means an intermediate join which has a higher cost leads to a lower cost in the end. It can either be the order
    of tuples satisfying the order required by \texttt{ORDER BY} or \texttt{GROUP BY}, or the order on a equality-join will enable cheaper
    sort-merge join in the next step. For example, assume we have a hash index on \texttt{R.a}, a B+ tree clustered on \texttt{W.c}, see 
    the following query: \begin{lstlisting}
  SELECT *
  FROM R, S, W
  WHERE R.a = S.b
    AND S.b = W.c;
    \end{lstlisting}
    For the step of joining \texttt{R} and \texttt{S}, an interesting order can be obtained by using a sorted-merge join, because for the 
    next step of joining with \texttt{W.c}, the tuples of \texttt{S.b} is already sorted and so is \texttt{W.c} using the B+ tree clustered
    index. Here's another example: \begin{lstlisting}
  SELECT SUM(R.a)
  FROM R, S, W
  WHERE R.a = S.b
    AND R.c = W.c
  GROUP BY S.b;
    \end{lstlisting}
    Now an interesting order is using sorted-merge-join on \texttt{R.a} and \texttt{S.b} because it will dramatically speed up the \texttt{GROUP BY}
    clause since every tuple is sorted on the \texttt{GROUP BY} field {S.b}.
     
\end{enumerate}
\end{document}
